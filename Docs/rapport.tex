\documentclass{article}

\usepackage[utf8]{inputenc}
\usepackage[francais]{babel} %% installer texlive-lang-french pour cela

% elements du titre
\title{Rapport projet 2, où l'on parle de Fouine}
\author{François Pitois et Simon Fernandez}
\date{\today}

\begin{document}

\maketitle

\section{Présentation}
Ce projet est un interprete, compilateur, machine à pile du langage Fouine.
Il a été programmé en CamL (souvent entre minuit et 3h) et aussi en partie
en Fouine (de 3h à 5h)

\section{Organisation du Code}
Le code est structuré de la manière suivante :
\begin{itemize}
\item  main : Point d’entrée du programme, gère les options passées `a l’execu-
table et appelle les différentes fonctions
\item  parser/lexer : Parsent l’entrée. Fournissent un arbre du code
\item  uncoment : fichier permettant de supprimer les commentaires d’un fichier
donné en entrée
\item types : concentre tous les types qui seront utilisés partout
\item  interprete : interprete de l’arborescence générée par lexer/parser, en vérifiant
les types
\item  zinc compiler : compile l’arborescence vers une suite de lexems qui seront
passés à la machine ZINC
\item  zinc machine : la machine ZINC qui va traiter les lexems, faire les calculs
et renvoyer le resultat. Elle est détaillée ici \cite{Zinc}
\item  machine types : les types et les tokens utilisés dans la machine à pile
\item de bruijn : traduit toute une arborescence pou utiliser les indices de De
Bruijn
\item convert : le convertisseur de Caml + extenstion vers Caml
\item exception.fou.ml : Fichier Fouine qui décrit les transformations de Caml + exception vers Caml
\end{itemize}

\section{Options possibles}
Nous avons implémenté les options suivantes : 
\begin{itemize}
\item Deux niveaux de debug : –verbose/-v –debug/-d
\item L’interprete qui vérifie aussi en direct la validité du typage
\item La machine ZINC, avec récursivité terminale
\item La traduction vers des indices de De Bruijn
\item L’export des lexems compilés vers un fichier exterieur
\item La traduction Fouine + Exceptions vers Fouine
\item Les fonctions prInt et feedback qui donnent des informations en cours d’interpretation
\end{itemize}

\section{Objets traités}
\begin{itemize}
\item Entiers
\item Booléens
\item Fonctions
\item Recursivité
\item Couples
\end{itemize}

\section{Spécificités de l'initerprete : les patterns}
Pour  gèrer  les  types  dans  l’interpreteur,  nous  avons  ajouté  l’idée  des  patterns. 
Les patterns permettent de traiter les fonctions, les paires, et le typage de variables. 
Ainsi, tout \texttt{Let \ldots = \ldots in} associe une expression à un pattern.
Donc  les  expressions  du  type \texttt{let a, b = 1, 2 in \ldots} fonctionnent  parfaitement
puisqu’on souhaite matcher la gauche du let avec la droite, et ainsi faire des définitions 
multiples

L’ajout des patterns a aussi permis de simplifier l’expressions du type
\texttt{let f x y z = \ldots}
Et éviter la surcharge d’un
\texttt{fun x -> fun y -> fun z -> \ldots }
Ces patterns ont aussi été utilisés par la ZINC

\section{Spécificités de la macine à pile : ZINC et De Bruijn}
\subsection{ZINC}
La machine se base sur l’implémentation ZINC, c’est à dire qu’on simplifie les
\texttt{let f x y z = \ldots} pour éviter d’avoir 3 RETURN de fonctions. On utilise
la récursivité terminale.
Exemple : En machine classique la fonction de $z$ rend la main à $y$
qui rend la main à $x$ qui rende enfin la main à l’appelant
Mais en recursion terminale, voyant que dès que $y$
aura récupéré la main il la rendra à $x$, on préfère rendre directement la 
main à l’appelant, à partir de $z$

\subsection{De Bruijn}
On utilise aussi la methode de De Bruijn pour les variables.
L’idée est d’éviter des comparaisons entre chaines de caracteres lors de la
recherche d’une variable dans l’environnement.
Tout se passe lors de la compilation, on garde trace de l’environnement à
chaque instant et on remplace les \texttt{ACCESS(x)} par des
\texttt{ACCESS(n)} où $n$ est l’indice de la variable dans l’environnement.
On a ainsi une recherche bien plus rapide et
une machine plus légère puisqu’elle ne traite plus du tout de chaine de caracteres.
Tout est nettoyé lors de la compilation


\bibliographystyle{plain}
\bibliography{biblio}

\end{document}









