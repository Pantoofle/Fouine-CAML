\documentclass{article}

\usepackage[utf8]{inputenc}
\usepackage[francais]{babel} %% installer texlive-lang-french pour cela

% ci-dessous: commenté car non offert sur les machines libre-service.
% décommentez si vous le souhaitez.
%\usepackage[french]{babel}

% pour compiler: 

% faire    pdflatex ex-rapport
% (si les references aux numeros de parties apparaissent comme des
% "?", recompiler une fois)

% la compilation de la bibliographie est davantage une "incantation":
% faire     bibtex ex-biblio
% puis      pdflatex ex-rapport (un nombre premier de fois)


% vous pouvez ensuite ouvrir le fichier ex-rapport.pdf




% elements du titre
\title{Rapport projet 2, où l'on parle de rongeurs}
\author{Madame Titi et Monsieur Toto}
\date{}


% definition de quelques macros, pour les maths
\newcommand{\litt}{\alpha}
\newcommand{\non}[1]{\overline{#1}}

\begin{document}

\maketitle

\section{Présentation}

Nous avons programmé \textbf{en D}, en interfaçant notre programme
avec des morceaux d'assembleur que nous avons écrits à la main, entre
3 et 4 heures du matin uniquement sinon ça ne compte pas.



Nous exposons à la partie~\ref{s:orga} comment notre programme est structuré.


Un peu de maths en \LaTeX: voici un exemple de formule~:
$$
\sum_{i\geq 0} \litt_1\lor\non{\litt_2}\lor\litt_4
$$
On remarque au passage que $\non{\non{\litt}}$ est pareil que $\litt$.

\section{Organisation du code}
\label{s:orga}

Le code est structuré de la manière suivante~:
\begin{itemize}
\item bli
\item bla
\item blo
\item Digression à propos des Mustélidés.
\end{itemize}

\section{Critique des performances}

On constate que blibla.


On est par ailleurs capable de citer des références, ainsi~: \cite{ProjInt16}.

\medskip

Pour citer une référence bibliographique, il faut insérer les
informations correspondantes au format BibTeX dans le fichier
\texttt{ex-biblio.bib}, et puis faire la citation en utilisant la
commande \verb+\cite{tititoto}+.

Ensuite, on compile de la manière suivante~:
\begin{enumerate}
\item \texttt{pdflatex ex-rapport}

et là il proteste, car il a vu une citation de \texttt{tititoto}, mais
ne sait pas à quoi cela fait référence

\item \texttt{bibtex ex-rapport}

et là il met ensemble les informations pour savoir engendrer
l'information correspondant à la citation de \texttt{tititoto}

\item \texttt{pdflatex ex-rapport}

et là il peut engendrer le fichier pdf, avec la bonne citation et la
bonne description dans les références
\end{enumerate}


\bibliographystyle{plain}
\bibliography{ex-biblio}

\end{document}
