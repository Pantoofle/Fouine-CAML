\documentclass{beamer}
\usepackage[francais]{babel}
\usepackage[utf8]{inputenc}
\usepackage{amsmath,amsfonts,amssymb}
\usepackage{xspace}
\usepackage{changepage}

\mode<presentation> {\usetheme{ENSLyon}}
\usecolortheme{ENSLyon_orange}

\setbeamersize{text margin left=35pt, text margin right=35pt} 
\usepackage{graphicx} % Allows including images
\usepackage{booktabs} % Allows the use of \toprule, \midrule and \bottomrule in tables


\title{PROJ 2}
\author{Pitois François et Fernandez Simon}
\institute{ENS Lyon}
%\medskip
\date{\today}

\begin{document}

\begin{frame}
\titlepage
\end{frame}

\section{Fonctions récursives}
\begin{frame}{Fonctions récursives}

  Gestion des fonctions récursives de l'interpréteur par évalution paresseuse :

  Dès que l'interprésteur voit un \texttt{let rec}, il le met dans l'environnement en tant que tel.

  L'interpréteur ne cherche à évaluer ce binding que quand il applique cette fonction récursive.

\end{frame}


\section{Machine ZINC}
\begin{frame}{Machine ZINC}
\begin{itemize}
	\item Récursion terminale 

	\texttt{let f x = 1 in \\ 
	let g y = f (x+1) in g 2}
		
	On détecte quand on peut éviter une étape quand on rend la main
\item Simplification des fonctions à plusieurs variables 
  
	\texttt{let f x y z = ... in f 1 2 3 ;;}
	
	\texttt{CLOS(CLOS(CLOS(...)))}
	
	On empile tous les arguments et on y accède en une seule fois
\end{itemize}	
\end{frame}


\section{Pattern}
\begin{frame}{Pattern}
  \texttt{let pattern = expr in expr ;;\\
	fun pattern -> expr ;;}

  Une pattern peut être :
  \begin{itemize}
  \item nom de variable
  \item \texttt{(pattern, pattern)}
  \item \texttt{pattern : type}
  \item \texttt{pattern -> pattern}
  \end{itemize}
\end{frame}


\section{Indice de De Bruijn}
\begin{frame}{Indice de De Bruijn}

  Simplification des noms de variables

  Éviter les comparaisons de chaines de caractères

  Indices dans l'environnement
 
  \texttt{let x = 3 in let y = 4 in x + y} 
	
	devient alors 
	
	\texttt{let 3 in let 4 in 0+1}

\end{frame}


\section{Conversion de fouine + blabla vers fouine}
\begin{frame}{Conversion de fouine + blabla vers fouine}

  Fait en fouine

  On utilise des noms de variable réservés pour repérer les zones du programme à modifier.
  
\end{frame}
\end{document}
