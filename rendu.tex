\section{PROJ2 : Rendu 2}\label{proj2-rendu-2}

Par François Pitois et Simon Fernandez

\section{Description}\label{description}

Ce projet est un interprete, compilateur, machine à pile du langage
Fouine.

\section{Contenu}\label{contenu}

\begin{itemize}
\tightlist
\item
  main.ml : fichier d'entrée. Fait l'appel aux differents modules :
  Parsing, Lexing, Interprete, etc\ldots{} Traite les options données
  pour appeler ce qu'il faut
\item
  lexer.mll : le lexer
\item
  parser.mly : le parser
\item
  types.ml : contient les differents types dans lesquels seront stockées
  les instructions lues par lexer/parser
\item
  interprete.ml : interprete de l'arborescence générée par lexer/parser,
  stockée dans les types décrits par types.ml
\item
  interprete\_aux.ml : fonctions utilitaires pour interprete. Sont juste
  utiles et rendent le code plus lisible.
\item
  compiler.ml : compile l'arborescence en une suite de lexems qui seront
  en suite traités par la machine à pile
\item
  machine.ml : machine à pile, prend une suite de lexems créés par
  compiler.ml, et les évalue, renvoit le haut de la pile apres
  l'execution
\item
  machine\_utils.ml : Un tas de fonctions utilitaires qui sont utilisées
  par la machine classique et par la ZINC
\item
  machine\_types.ml : Contient les lexems de la machine et ceux des
  différentes stacks
\item
  zinc\_compiler.ml : le compilateur de la version ZINC de la machine
\item
  zinc\_machine.ml : la machine de la version ZINC
\item
  de\_bruijn.ml : applique la transformation de De bruijn
\item
  convert.ml : le convertisseur de Caml + extenstion vers Caml
\item
  exception.fou.ml : Fichier fouine qui décrit les transformations de
  Caml + exception vers Caml
\item
  Makefile : le Makefile du projet
\item
  README.md : ce document
\item
  exemples.fou.ml : Code fouine contenant de nombreux exemples sur les
  differentes possibilités de l'interprete et de la machine
\end{itemize}

\section{Utilisation}\label{utilisation}

Pour compiler les sources, le Makefile se charge de tout. Il faut juste
taper make et tout est créé. L'executable s'appelle fouine

Le resultat de l'execution est soit un entier, soit une fonction. Dans
ce dernier cas, l'interprete affichera le code CamL correspondant, et la
machine à pile affichera la fonction sous la forme d'une suite de lexems
correspondants à la version compilée de la fonction

\subsection{Options}\label{options}

\begin{itemize}
\tightlist
\item
  -d ou --debug : Rend l'execution verbeuse. Seront affichés differentes
  informations intermediaires sur ce qu'il se passe lors de l'execution.
\item
  -m ou --machine : Compile le programme donné et l'execute avec la
  machine à pile
\item
  -z ou --zinc : Compile le programme et le lance sur la machine ZINC
\item
  -i ou --interm toto.code : Compile le programme et stocke la suite de
  lexems dans toto.code sans essayer de l'executer
\item
  -a ou --all : Lance l'interpreteur ET la machine à pile ET la ZINC sur
  la même entrée, pour comparer les sorties
\item
  -E : Pour lancer la traduction vers fouine sans exception à partir de
  Fouine avec exception. (MAIS ne marche pas avec les fonction
  récursives)
\end{itemize}

\section{Parties non traitées :}\label{parties-non-traituxe9es}

\begin{itemize}
\tightlist
\item
  Recursivité sur des variables dans la machine à pile
\item
  Le moins unaire
\item
  L'option -R et -RE
\end{itemize}

\section{Options supplémentaires}\label{options-suppluxe9mentaires}

\begin{itemize}
\tightlist
\item
  En plus du prInt qui affiche la valeur de l'int, nous avons ajouté
  feedback, qui affiche toute la valeur demandée (donc même si c'est une
  fonction ou une reference), sous le même format que ce qui est renvoyé
  pour le resultat
\item
  Une option -debug pour encore plus de détail. Bien plus verbeux que
  -verbose
\end{itemize}

\section{Options et langage de la Machine à
pile}\label{options-et-langage-de-la-machine-uxe0-pile}

\subsection{Possibilités}\label{possibilituxe9s}

\begin{itemize}
\item
  Arithmétique classique
\item
  If then else
\item
  let \ldots{} in
\item
  prInt et Feedback
\item
  Fonctions
\item
  References d'entiers et de fonctions
\item
  Fonctions recursives
\item
  Exceptions
\item
  Indices de De Bruijn : OK sur la classique et sur ZINC
\end{itemize}

\subsection{Reste à faire}\label{reste-uxe0-faire}

\begin{itemize}
\tightlist
\item
  Tableaux
\item
  Comparaison d'efficacité entre la classique et ZINC
\end{itemize}

\subsection{Lexems}\label{lexems}

\begin{itemize}
\item
  CONST\_INT(n) : Represente une constante entière.
\item
  ACCESS(n) : Va chercher le n-ème terme en memoire et le met sur le
  haut de la pile
\item
  GRAB : Prend le haut de la pile et le met sur le haut de
  l'environnement
\item
  CLOS(code) : Representation d'une fonction : la suite de lexems à
  egffectuer pour appliquer la fonction. La variable est la première sur
  l'environnement
\item
  OP(c) : Une opération arithmétique ou booléenne, sous la forme d'un
  seul caractere. Sont acceptés : +, -, *, /, =, \& (le ``et'' logique),
  \textbar{} (le ``ou'' logique), \textless{}, \textgreater{}
\item
  APPLY : Appel la fonction qui est sur le dessus de la pile sur
  l'argiment qui est en 2e position sur la pile.
\item
  PRINT : Affiche la valeur entière du dessus de la pile, sans la
  dépiler
\item
  FEEDBACK : Affiche la valeur du dessus de la pile, sans la dépiler
\item
  ENDLET : Supprime la valeur qui est sur le haut de l'environnement
\item
  IF : Dépile le haut de la pile, si c'est true, execute jusqu'au ELSE
  et ensuite saute tout jusqu'au ENDIF. Si c'est false, on execute
  uniquement la portion entre ELSE et ENDIF
\item
  RETURN : Pour revenir d'un appel de fonction. Va chercher sur la pile
  le code et l'environnement qui étaient en attente lors du APPLY pour
  continuer l'execution.
\item
  REF : Comme GRAB mais stocke la valeur sous la forme d'une reference
\item
  DEREF : Bang sur la référence qui est sur le haut de la pile
\item
  ASSIGN(n) : Assigne la valeur qui est sur le haut de la pile à la
  reference x qui est déjà dans l'environnement.
\item
  CLOS\_R(code) : Comme CLOS, mais pour des fontions recursives. Si lors
  du APPLY c'est une fonction recursive qu'on se retrouve à appeler, on
  la push dans son propre environnement avant de l'executer. Pout les
  indices de De Bruijn, la variable est en position 2 sur la pile et la
  fonction en position 1
\item
  SEQ : Pour délimiter des séquences : dépile le haut de la stack.
\item
  TRY : Dépile la fonction sur la stack et l'ajoute sur la pile
  d'erreurs
\item
  RAISE : Dépile la fonction sur la pile d'erreur, effectue son calcul
  et reprend le code là ou il faut
\item
  ENDTRY : Dépile une fonction de la pile d'erreurs. Arrive quand rien
  n'a été RAISE
\end{itemize}

Pour ZINC - TAILAPPLY : Appel de apply avec recursivité terminale -
PUSHMARK : Met une marque sur la pile pour délimiter les arguments
passés à une fonction sur la pile d'arguments

\section{Repartition du travail}\label{repartition-du-travail}

\begin{itemize}
\tightlist
\item
  main : Simon
\item
  lexer/parser : François
\item
  types : François et Simon
\item
  interprete : François
\item
  compiler : Simon
\item
  machine : Simon
\item
  zinc : Simon
\item
  De Bruijn : Simon
\item
  convertisseur : François
\item
  Exemples : François
\item
  Readme : Simon
\end{itemize}
